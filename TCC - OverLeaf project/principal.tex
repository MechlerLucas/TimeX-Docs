%%%%%%%%%%%%%%%%%%%%%%%%%%%%%%%%%%%%%%%%%%%%%%%%%%%%%%%%%%%%%%%%%%%%%%%%%%%%%%%%%%%%%%%%%%%%%%%%%%%%%%%
%%%%%%%%%%%%%% Template de Artigo Adaptado para Trabalho de Conclusão de Curso - SI Contagem - PUCMINAS %%%%%%%%%%%%%%%%%%%%%%%%
%% codificação UTF-8 - Abntex - Latex -  							     %%
%% Autor da primeira versão:    Fábio Leandro Rodrigues Cordeiro  (fabioleandro@pucminas.br)                            %% 
%% Co-autor da primeira versão: Prof. João Paulo Domingos Silva, Harison da Silva e Anderson Carvalho                   %%
%% Revisores normas NBR (Padrão PUC Minas) da primeira versão: Helenice Rego Cunha e Prof. Theldo Cruz                  %%
%% Versão: 1.1     18 de dezembro 2015                     	                                     %%
%%%%%%%%%%%%%%%%%%%%%%%%%%%%%%%%%%%%%%%%%%%%%%%%%%%%%%%%%%%%%%%%%%%%%%%%%%%%%%%%%%%%%%%%%%%%%%%%%%%%%%%


\documentclass[a4paper,12pt]{article}
\usepackage{times}
\usepackage{abakos}  %pacote com padrão da Abakos baseado no padrão da PUC
%%%%%%%%%%%%%%%%%%%%%%%%%%%
%Capa da revista
%%%%%%%%%%%%%%%%%%%%%%%%%%

%\setcounter{page}{80} %iniciar contador de pagina de valor especificado
\newcommand{\monog}{TimeX - Aplicativo de gerenciamento de tempo e organização de tarefas focado em desafios de organização associados ao TDAH}
%\newcommand{\monogES}{Article template Institute of Mathematical Sciences and Informatics}
\newcommand{\tipo}{Artigo}  % Especificar a seção tipo do trabalho: Artigo, Resumo, Tese, Dociê etc
%\newcommand{\origem}{Brasil}
%\newcommand{\editorial}{Belo Horizonte, p. 01-11, nov. 2024}  % p. xx-xx – páginas inicial-final do artigo
\newcommand{\editorial}{}  
\newcommand{\lcc}{\scriptsize{Licença Creative Commons Attribution-NonCommercial-NoDerivs 3.0 Unported}}

%%%%%%%%%%%%%%%%%INFORMAÇÕES SOBRE AUTOR PRINCIPAL %%%%%%%%%%%%%%%%%%%%%%%%%%%%%%%
\newcommand{\AutorA}{Lucas Mechler Fernandes}
\newcommand{\funcaoA}{}
\newcommand{\emailA}{\href{mailto:xxx@sga.pucminas.br}{lucas.mechler@sga.pucminas.br}}
\newcommand{\cursA}{Aluno(a) do Programa de Graduação em Sistemas de Informação}

\newcommand{\AutorB}{Luciana Mara Freitas Diniz}
\newcommand{\funcaoB}{}
\newcommand{\emailB}{\href{mailto:xxx@sga.pucminas.br}{lmfdiniz@sga.pucminas.br}}
\newcommand{\cursB}{Professor(a) do Programa de Graduação em Sistemas de Informação}
% 
% Definir macros para o nome da Instituição, da Faculdade, etc.
\newcommand{\univ}{Pontifícia Universidade Católica de Minas Gerais}

\newcommand{\keyword}[1]{\textsf{#1}}

%Para as URLS não ultrapassarem a margem
\usepackage{url}
\makeatletter
\g@addto@macro{\UrlBreaks}{\UrlOrds}
\makeatother

\usepackage{booktabs}
\usepackage{graphicx}
\usepackage{subcaption}

\usepackage{adjustbox} % reduzir tamanho figuras
\setlength{\headheight}{22.38pt}

\usetikzlibrary{arrows,shapes,positioning,shadows,trees}

\tikzset{
  basic/.style  = {draw, text width=3cm, drop shadow, font=\sffamily, rectangle},
  root/.style   = {basic, rounded corners=2pt, thin, align=center,
                   fill=blue!30},
  level 2/.style = {basic, rounded corners=6pt, thin,align=center, fill=green!30,
                   text width=8em},
  level 3/.style = {basic, thin, align=left, fill=pink!60, text width=6.5em}
}

\usepackage{tikz}
\usetikzlibrary{arrows.meta, positioning, shapes.geometric}


\begin{document}
% %%%%%%%%%%%%%%%%%%%%%%%%%%%%%%%%%%
% %% Pagina de titulo
% %%%%%%%%%%%%%%%%%%%%%%%%%%%%%%%%%%

\begin{center}
\includegraphics[scale=0.2]{figuras/brasao.jpg} \\
PONTIFÍCIA UNIVERSIDADE CATÓLICA DE MINAS GERAIS \\
Instituto de Ciências Exatas e de Informática

% \vspace{1.0cm}

\end{center}

\begin{center}
 \vspace{0cm} {
 \singlespacing \Large{\monog \symbolfootnote[1]{Artigo apresentado ao Instituto de Ciências Exatas e Informática da Pontifícia Universidade Católica de \linebreak Minas Gerais, campus Contagem, como pré-requisito parcial para obtenção do título de Bacharel em Sistemas de Informação.} \\ }
 % \normalsize{\monogES}
 }
 \end{center}

\vspace{1.0cm}

\begin{flushright}
\singlespacing 
\normalsize{\AutorA \footnote{\funcaoA \cursA \,-- \emailA . }} \\
\normalsize{\AutorB \footnote{\funcaoB \cursB \,-- \emailB . }} \\
%\normalsize{\AutorC \footnote{\funcaoC \cursC \,-- \emailC . }} \\
%\normalsize{\AutorD \footnote{\funcaD \cursD \,-- \emailD . }} \\
\end{flushright}
\thispagestyle{empty}

\vspace{1.0cm}

\selectlanguage{brazilian}
\begin{abstract}
\noindent
O Transtorno de Déficit de Atenção e Hiperatividade (TDAH) afeta significativamente a organização pessoal e o cumprimento de tarefas cotidianas. Embora existam diversos aplicativos de agenda e despertadores, muitos apresentam interfaces densas ou complexas, o que pode dificultar o uso por pessoas que enfrentam desafios de foco e organização descritos na literatura. Nesse contexto, este trabalho propõe o desenvolvimento de um aplicativo móvel simples, combinando funções de despertador e checklist de tarefas, com design limpo e intuitivo. O objetivo é oferecer uma solução que auxilie a organização pessoal por meio de recursos visuais claros e de baixa sobrecarga cognitiva. A metodologia envolveu revisão bibliográfica, definição de requisitos essenciais, implementação de funcionalidades como lembretes e categorização por cores, além da criação de um protótipo avaliado por heurísticas de usabilidade. Conclui-se que o protótipo apresenta potencial para apoiar usuários que enfrentam dificuldades de organização, servindo como base para estudos futuros e aprimoramentos.
\\\textbf{\keyword{Palavras-chave:}} TDAH; organização pessoal; despertador; aplicativo móvel; usabilidade.
\end{abstract}

\pagebreak

%%%%%%%%%%%%%%%%%%%%%%%%%%%%%%%%%%%%%%%%%%%%%%%%%%%%%%%%%
\selectlanguage{english}
\begin{abstract}
\noindent
Attention Deficit Hyperactivity Disorder (ADHD) can significantly affect personal organization and the execution of daily tasks. Although several scheduling and alarm applications are available, many of them offer dense or complex interfaces that may hinder their use by individuals who face focus and organizational challenges described in the literature on ADHD. In this context, this work proposes the development of a simple mobile application that combines alarm functions with task checklists, designed with a clean and intuitive interface. The objective is to provide a tool that assists personal organization through clear visual cues and reduced cognitive load. The methodology included a literature review, definition of essential requirements, implementation of features such as reminders and color categorization, and the creation of a prototype evaluated using usability heuristics. The results indicate that the prototype has the potential to support users who experience organizational difficulties, serving as a basis for future studies and improvements.
\\\textbf{\keyword{Keywords:}} ADHD; personal organization; alarm; mobile application; usability.
\end{abstract}


\selectlanguage{brazilian}
 \onehalfspace  % espaçamento 1.5 entre linhas
 \setlength{\parindent}{1.25cm}

%%%%%%%%%%%%%%%%%%%%%%%%%%%%%%%%%%%%%%%%%%%%%%%%%
%% INICIO DO TEXTO
%%%%%%%%%%%%%%%%%%%%%%%%%%%%%%%%%%%%%%%%%%%%%%%%%

%%%%%%%%%%%%%%%%%%%%%%%%%%%%%%%%%%%%%%%%%%%%%%%%%%%%%%%%%%%%%%%%%%%%%%%%%%%%%%%%%%%%%%%%%%%%%%%%%%%%%%%%%%%%%%%%%%%%%%%%%%%%%%%%
%%%%%%%%%%%%%% Template de Artigo Adaptado para Trabalho de Conclusão de Curso - SI Contagem - PUCMINAS                       %%
%% codificação UTF-8 - Abntex - Latex -  							                                                          %%
%% Autor da primeira versão:    Fábio Leandro Rodrigues Cordeiro                                                              %% 
%% Co-autores da primeira versão: Prof. João Paulo Domingos Silva, Harison da Silva e Anderson Carvalho		                  %%
%% Revisores normas NBR (Padrão PUC Minas) da primeira versão: Helenice Rego Cunha e Prof. Theldo Cruz                        %%
%% Versão: 1.1     18 de dezembro 2015                                                                                        %%
%%%%%%%%%%%%%%%%%%%%%%%%%%%%%%%%%%%%%%%%%%%%%%%%%%%%%%%%%%%%%%%%%%%%%%%%%%%%%%%%%%%%%%%%%%%%%%%%%%%%%%%%%%%%%%%%%%%%%%%%%%%%%%%%
\section{\esp Introdução} 

O Transtorno de Déficit de Atenção e Hiperatividade (TDAH), conforme apud de \citeonline{carolo2009} sobre Barkley, é um transtorno neurobiológico caracterizado por dificuldades nas funções executivas, como planejamento, organização e autorregulação, o que frequentemente resulta em prejuízos acadêmicos, profissionais e sociais. Nesse contexto, a área de aplicativos móveis e ferramentas digitais de apoio vem se consolidando como uma alternativa promissora para auxiliar pessoas com esse perfil, oferecendo recursos que podem apoiar a organização pessoal de maneira prática e acessível.

Observa-se que muitos aplicativos populares de produtividade apresentam excesso de funcionalidades, menus extensos e elementos visuais que podem contribuir para a perda de foco e abandono do uso. Estudos apontam que interfaces simples, limpas e com menos estímulos tendem a favorecer o uso contínuo por pessoas com dificuldades de atenção sustentada, conforme discutido por \citeonline{knouse2022}, \citeonline{almeida2024} e \citeonline{ferreira2023}. Assim, este trabalho parte da premissa de que soluções visuais minimalistas e processos reduzidos podem ser potencialmente mais adequados para usuários que enfrentam desafios de organização diária, em alinhamento com recomendações presentes na literatura.

A escolha deste tema justifica-se pela relevância social e acadêmica da proposta. Do ponto de vista social, observa-se que a maioria dos aplicativos populares de produtividade não é projetada para públicos neurodivergentes, e frequentemente apresenta excesso de menus, funções avançadas e telas densas. Para muitas pessoas com TDAH, esse excesso de elementos funciona como um fator de distração, contribuindo para a perda de foco e o abandono da ferramenta. Assim, a simplicidade torna-se não apenas desejável, mas uma estratégia essencial para reduzir a carga cognitiva durante o uso. Do ponto de vista acadêmico, este trabalho busca evidenciar como aplicações minimalistas podem ser potencialmente mais adequadas para usuários que apresentam dificuldades de foco e organização descritas na literatura sobre TDAH.

Em uma análise exploratória realizada em cinco aplicativos populares de gerenciamento de tarefas, observou-se que nenhum deles oferece recursos como reforços visuais por cores para diferentes níveis de prioridade ou alertas múltiplos para uma mesma tarefa, funcionalidades apontadas na literatura como benéficas para esse público.

Portanto, este estudo pretende oferecer uma alternativa inspirada nas dificuldades relatadas na literatura e reforçadas por feedback exploratório obtido durante o desenvolvimento, como simplicidade visual, feedbacks imediatos, baixo número de etapas por ação, organização por cores e categorias claras e redução da sobrecarga cognitiva durante o uso do aplicativo.

Nesse sentido, o objetivo geral deste trabalho é desenvolver um aplicativo simples de organização pessoal e gerenciamento de tempo, fundamentado em princípios de simplicidade, clareza visual e redução da carga cognitiva, elaborado com base nas dificuldades de organização e foco descritas na literatura sobre TDAH, buscando compreender como estratégias minimalistas podem auxiliar o usuário ao diminuir distrações e tornar a navegação mais direta.

São objetivos específicos deste trabalho:
(i) realizar uma revisão bibliográfica sobre TDAH e estratégias de organização pessoal; (ii) identificar, a partir da literatura , os requisitos funcionais e não funcionais para o aplicativo; (iii) implementar funcionalidades essenciais como alarmes adaptados, lembretes simples e categorização por cores, priorizando a simplicidade e evitando elementos que possam gerar sobrecarga cognitiva; (iv) desenvolver um protótipo navegável do aplicativo; e (v) realizar uma avaliação preliminar utilizando de plataforma de avaliação de usabilidade.

%%%%%%%%%%%%%%%%%%%%%%%%%%%%%%%%%%%%%%%%%%%%%%%%%%%%%%%%%%%%%%%%%%%%%%%%%%%%%%%%%%%%%%%%%%%%%%%%%%%%%%%%%%%%%%%%%%%%%%%%%%%%%%%%

\section{\esp Referencial Teórico}

Nesta seção são apresentados os principais conceitos que fundamentam este trabalho. Serão discutidos os aspectos relacionados ao Transtorno de Déficit de Atenção e Hiperatividade (TDAH), as tecnologias assistivas digitais voltadas à organização pessoal e, por fim, os princípios de usabilidade aplicados ao desenvolvimento de aplicativos claros e funcionais. O objetivo é fornecer uma base teórica que permita compreender como fatores psicológicos, tecnológicos e de design se conectam na criação desoluções digitais voltadas às dificuldades de organização associadas ao TDAH.

\subsection{\esp Neurodivergência e TDAH}

O conceito de neurodivergência refere-se à variação natural do funcionamento neurológico humano, englobando condições como Transtorno do Espectro Autista (TEA), dislexia e TDAH. Em vez de ser compreendida apenas como um déficit ou patologia, a neurodivergência enfatiza a diversidade cognitiva como parte da condição humana. Dentro desse espectro, o TDAH é um transtorno do neurodesenvolvimento caracterizado por sintomas persistentes de desatenção, impulsividade e, em alguns casos, hiperatividade \citeonline{carolo2009}. Esses sintomas se manifestam desde a infância e podem permanecer ao longo da vida adulta, gerando impactos significativos na vida acadêmica, social e profissional dos indivíduos.

As dificuldades mais recorrentes no TDAH estão associadas ao controle inibitório, à memória de trabalho e à autorregulação emocional, habilidades fundamentais para planejar, organizar e executar atividades diárias \citeonline{barkley2008}. Por isso, é comum que pessoas com TDAH enfrentem problemas em cumprir prazos, gerenciar rotinas, manter o foco em tarefas longas e lidar com a procrastinação. Estudos recentes reforçam que essas limitações não se explicam apenas pela falta de esforço ou disciplina, mas decorrem de diferenças estruturais e funcionais no cérebro, principalmente no córtex pré-frontal \citeonline{knouse2022}.

Segundo dados nacionais, a prevalência do Transtorno do Déficit de Atenção com Hiperatividade (TDAH) no Brasil é estimada em aproximadamente 7,6\% entre crianças e adolescentes de 6 a 17 anos, passando para cerca de 5,2\% na faixa etária de 18 a 44 anos e 6,1\% nos indivíduos com 44 anos ou mais \citeonline{conitec2022}. Em âmbito global, a Associação Brasileira do Déficit de Atenção estima que a prevalência situa-se entre 5\% e 8\% da população, o que reforça a relevância social e acadêmica de desenvolver ferramentas e estratégias de apoio voltadas a esse público \citeonline{abda2022}.  

Diante dessas dificuldades, torna-se necessário recorrer a soluções externas de apoio, como as tecnologias assistivas digitais, apresentadas a seguir.

No entanto, pesquisas recentes apontam que muitos aplicativos populares de produtividade apresentam excesso de funcionalidades, menus complexos e estímulos visuais que podem prejudicar especialmente usuários neurodivergentes. \citeonline{knouse2022} destacam que pessoas com TDAH tendem a abandonar ferramentas digitais quando a interface contém elementos supérfluos, pois a sobrecarga cognitiva gerada dificulta o foco e a continuidade do uso. De forma semelhante, \citeonline{almeida2024} argumentam que interfaces minimalistas, com poucos elementos por tela e fluxos curtos, favorecem a manutenção da atenção e reduzem distrações. \citeonline{ferreira2023} reforça que a simplicidade não é apenas uma preferência estética, mas um requisito essencial para reduzir a carga cognitiva e promover maior clareza durante o uso.

\subsection{\esp Tecnologias assistivas digitais}

As tecnologias assistivas compreendem um conjunto de recursos, serviços e dispositivos que buscam ampliar a autonomia e a qualidade de vida de pessoas com deficiências ou condições específicas. No Brasil, a Lei nº 13.146/2015 (Estatuto da Pessoa com Deficiência) define tecnologia assistiva como todo recurso, metodologia ou prática que possibilite compensar limitações funcionais, promovendo inclusão e participação social. No contexto do TDAH, essas tecnologias assumem uma dimensão particular, pois o foco recai sobre apoiar a organização pessoal, a gestão do tempo e a manutenção da atenção.

O avanço dos dispositivos móveis e a popularização dos smartphones ampliaram significativamente as possibilidades de intervenção digital. Aplicativos de produtividade, agendas eletrônicas e sistemas de lembretes passaram a ser utilizados como ferramentas compensatórias para pessoas com TDAH. Segundo \citeonline{knouse2022}, soluções digitais baseadas em terapia cognitivo-comportamental têm mostrado potencial para reduzir sintomas e aumentar a adesão a rotinas. \citeonline{ferreira2023} destaca que a simplificação das interfaces é fundamental para evitar sobrecarga cognitiva, enquanto Ramos, Ferreira e \citeonline{ramos2024} exploram a gamificação como meio de tornar a experiência mais motivadora. Já \citeonline{almeida2024} reforçam a importância de elementos visuais limpos e acessíveis.

Além dos aplicativos voltados especificamente ao TDAH, métodos tradicionais de organização também foram adaptados para o ambiente digital. O GTD \textit{(Getting Things Done)}, de \citeonline{allen2001}, e a Técnica Pomodoro, de \citeonline{cirillo2006}, são amplamente utilizados para melhorar a produtividade, embora apresentem limitações quando aplicados a pessoas com TDAH, devido à sua complexidade ou rigidez. Por esse motivo, adaptações como checklists simplificados, múltiplos alarmes e lembretes progressivos têm sido apontados como estratégias mais adequadas.

Assim, as tecnologias assistivas digitais configuram-se como um campo fértil para a inovação, especialmente quando direcionadas às dificuldades de organização associadas ao TDAH. Ao combinar princípios de produtividade com recursos tecnológicos adaptados, é possível criar soluções que ampliem a autonomia e a motivação dos usuários.

\subsection{\esp Usabilidade em aplicações para TDAH}

O conceito de usabilidade está associado à facilidade de uso de um sistema para que os usuários alcancem seus objetivos de forma eficaz, eficiente e satisfatória \citeonline{nielsen1994}.Ele propôs dez heurísticas fundamentais de usabilidade, que orientam o design centrado no usuário, a serem detalhadas nos resultados:

No contexto do TDAH, essas heurísticas ganham importância especial. Por exemplo, o design estético e minimalista e o reconhecimento em vez de memorização ajudam a reduzir a sobrecarga cognitiva; a prevenção de erros e o feedback imediato contribuem para manter a atenção do usuário; e a consistência visual favorece a previsibilidade das ações, reduzindo distrações e frustrações.

Assim, o desenvolvimento de aplicações voltadas a pessoas com TDAH deve alinhar-se a essas heurísticas, garantindo interfaces simples, coerentes e acessíveis.

Pesquisas recentes indicam que o design de interfaces deve privilegiar a simplicidade, a consistência visual e a clareza de instruções \citeonline{almeida2024}. No contexto do TDAH, recomenda-se evitar excesso de estímulos, utilizar cores para organizar prioridades e criar lembretes múltiplos para uma mesma atividade. A remoção de elementos que pouco contribuem para a eficácia, como o botão de “soneca” em alarmes, também se mostra relevante, uma vez que pode induzir à procrastinação. Essas práticas visam reduzir a carga cognitiva do usuário, facilitando a interação com o aplicativo.

Portanto, observa-se que a integração entre conceitos de neurodivergência, tecnologias assistivas e usabilidade fornece um alicerce sólido para o desenvolvimento de soluções digitais voltadas a pessoas com TDAH. O domínio desses fundamentos permite compreender as especificidades do público-alvo e direcionar esforços de design e implementação para aplicativos mais claros, funcionais e alinhados às necessidades descritas na literatura.

Esses princípios de simplicidade, clareza e adaptação às limitações cognitivas orientam diretamente a proposta deste trabalho, que busca oferecer uma solução digital funcional e inclusiva para pessoas com TDAH.

%%%%%%%%%%%%%%%%%%%%%%%%%%%%%%%%%%%%%%%%%%%%%%%%%%%%%%%%%%%%%%%%%%%%%%%%%%%%%%%%%%%%%%%%%%%%%%%%%%%%%%%%%%%%%%%%%%%%%%%%%%%%%%%%

\section{\esp Trabalhos Relacionados}

O estudo de \citeonline{knouse2022} teve como objetivo avaliar a usabilidade e viabilidade do aplicativo móvel Inflow, baseado em terapia cognitivo-comportamental, para adultos com TDAH. A pesquisa foi conduzida em formato de estudo aberto de sete semanas, com 240 participantes recrutados online que responderam avaliações em diferentes pontos de tempo. Foram coletados relatos dos próprios usuários sobre sintomas, uso do aplicativo e satisfação com a experiência. Os resultados indicaram boa aceitação, uso frequente e percepção de redução nos sintomas e prejuízos relacionados ao TDAH, demonstrando viabilidade e recomendando novos estudos clínicos controlados. Tal proposta se aproxima deste trabalho pelo foco em soluções móveis voltadas a pessoas com TDAH, mas se diferencia por ter como base a terapia cognitivo-comportamental, enquanto o presente estudo foca na organização pessoal por meio de despertadores inteligentes e checklists.

A pesquisa de \citeonline{ferreira2023} buscou propor o desenvolvimento de uma aplicação para auxiliar pessoas com TDAH na gestão de tarefas. A metodologia envolveu revisão bibliográfica, análise de necessidades dos usuários e a prototipagem de uma aplicação digital com fluxos de interação simplificados. O resultado foi a elaboração de um protótipo inicial, que mostrou potencial para facilitar a rotina de usuários com TDAH, embora o autor ressalte a necessidade de avaliações de usabilidade em etapas posteriores. Essa proposta se aproxima do presente trabalho por priorizar a simplicidade e funcionalidade, mas difere por não incluir recursos de alarmes inteligentes, que aqui são considerados centrais.

O estudo de \citeonline{almeida2024} teve como objetivo analisar como estratégias de design podem impulsionar a criação de aplicativos para TDAH. Para isso, foram utilizados métodos de revisão teórica, análise de recursos em aplicativos já existentes e aplicação de princípios de design centrado no usuário. Os resultados apontaram que interfaces limpas, estímulos visuais controlados e navegação simplificada são fatores determinantes para adesão e eficácia das ferramentas digitais. A semelhança com este trabalho está no foco na experiência do usuário e na simplicidade do design, mas difere-se por manter o design como objeto principal de análise, enquanto aqui a ênfase é na implementação prática de um aplicativo funcional.

Já o trabalho de \citeonline{ramos2024} teve como objetivo explorar a aplicação da gamificação como recurso auxiliar no tratamento de jovens e adultos com TDAH. A metodologia incluiu revisão bibliográfica e pesquisa de campo com 50 participantes, que responderam questionários sobre suas dificuldades e preferências. A partir desses dados, foi proposto o protótipo do aplicativo Quest Mind, que utiliza mecânicas de jogos para estimular motivação, organização e adesão ao tratamento. Os resultados apontaram boa aceitação da proposta e potencial para impactos positivos no tratamento. Essa iniciativa se assemelha ao presente trabalho pela busca de soluções digitais voltadas à organização de pessoas com TDAH, mas se distingue pela ênfase em gamificação como estratégia principal, enquanto este estudo propõe o uso de despertadores adaptados e checklists como instrumentos centrais.

%%%%%%%%%%%%%%%%%%%%%%%%%%%%%%%%%%%%%%%%%%%%%%%%%%%%%%%%%%%%%%%%%%%%%%%%%%%%%%%%%%%%%%%%%%%%%%%%%%%%%%%%%%%%%%%%%%%%%%%%%%%%%%%%

\section{Metodologia}

A metodologia adotada neste trabalho é de natureza aplicada e de desenvolvimento. O objetivo principal foi criar um aplicativo web funcional, denominado \textit{TimeX}, voltado à organização e gerenciamento de tempo de pessoas com sintomas ou diagnóstico de TDAH.

\subsection{Classificação da pesquisa}

O estudo é classificado como aplicado, pois busca gerar uma solução prática fundamentada em bases teóricas. É também exploratório e descritivo, ao investigar as necessidades do público-alvo e descrever o processo de desenvolvimento da aplicação.

\subsection{Etapas de desenvolvimento}

O processo de desenvolvimento foi dividido em etapas principais:

\begin{enumerate}[label=\alph*)]
    \item Revisão bibliográfica: Pesquisa em bases acadêmicas como Scielo, Google Scholar e ACM Digital Library, com foco em TDAH, tecnologias assistivas e usabilidade \cite{barkley2008,nielsen1994,norman2013,almeida2024,ferreira2023}.
    
    \item Definição de requisitos: Os requisitos funcionais e não funcionais foram definidos com base na literatura e na análise de aplicativos similares, priorizando simplicidade, clareza e redução de distrações.
    
    \item Design da interface: A interface foi projetada segundo os princípios de usabilidade de \citeonline{nielsen1994} e design centrado no usuário de \citeonline{norman2013}. Adotou-se o conceito \textit{mobile-first}, com layout limpo e intuitivo. Utilizou-se Tailwind CSS \cite{tailwindcss}, Shadcn/UI, Framer Motion e Lucide React.
    
    \item Implementação: O frontend foi desenvolvido em React.js \cite{react}, com gerenciamento de estado por React Query e navegação com React Router DOM. O backend foi construído localmente em Node.js com Express.js \cite{expressjs}, utilizando arquivos JSON para persistência de dados e estrutura de rotas REST.
    
    \item Testes e validação: Foram realizados testes exploratórios para verificar o funcionamento das rotas, da interface e das notificações do navegador. As funcionalidades foram avaliadas conforme os princípios de usabilidade e simplicidade identificados na literatura.
\end{enumerate}

Além disso, foi obtido um feedback exploratório de um usuário diagnosticado com TDAH, que testou o protótipo de forma informal, contribuindo com observações qualitativas consideradas posteriormente na discussão dos resultados.


\subsection{Ambiente de desenvolvimento}

O ambiente de desenvolvimento foi configurado localmente utilizando o Visual Studio Code e o Node.js. As dependências foram gerenciadas via NPM e o projeto estruturado em pastas separadas para frontend e backend. A Tabela~\ref{tab:tecnologias} apresenta as principais ferramentas utilizadas.

\begin{table}[H]
\centering
\caption{Tecnologias utilizadas no desenvolvimento do aplicativo TimeX}
\label{tab:tecnologias}
\renewcommand{\arraystretch}{1.3}

\footnotesize

\begin{tabular}{p{4cm} p{10cm}}
\hline
Tecnologia/Ferramenta & Descrição resumida \\
\hline
React.js & Biblioteca JavaScript para construção de interfaces de usuário. \\
\hline
Tailwind CSS & Framework CSS utilitário para estilização responsiva. \\
\hline
Shadcn/UI & Conjunto de componentes acessíveis baseados em Tailwind CSS. \\
\hline
Framer Motion & Biblioteca de animação para React. \\
\hline
Lucide React & Pacote de ícones vetoriais leves. \\
\hline
React Router DOM & Gerenciador de rotas e navegação. \\
\hline
React Query & Gerenciamento de estado e cache de requisições assíncronas. \\
\hline
date-fns & Manipulação e formatação de datas. \\
\hline
Express.js & Framework para criação de APIs REST locais em Node.js. \\
\hline
Node.js & Ambiente de execução JavaScript no servidor. \\
\hline
Visual Studio Code & Ambiente de desenvolvimento integrado (IDE). \\
\hline
\end{tabular}
\end{table}



%%%%%%%%%%%%%%%%%%%%%%%%%%%%%%%%%%%%%%%%%%%%%%%%%%%%%%%%%%%%%%%%%%%%%%%%%%%%%%%%%%%%%%%%%%%%%%%%%%%%%%%%%%%%%%%%%%%%%%%%%%%%%%%%

\section{Resultados e Discussão}

Esta seção apresenta os principais resultados obtidos com o desenvolvimento do aplicativo \textit{TimeX}, descrevendo sua implementação, suas características voltadas à redução da carga cognitiva e a avaliação heurística realizada a partir das diretrizes de \citeonline{nielsen1994}.

\subsection{Implementação do TimeX}

O \textit{TimeX} foi desenvolvido como um aplicativo web com arquitetura modular composta por uma interface em React.js e um backend local em Node.js e Express.js. A persistência de dados é realizada por meio de arquivos JSON, garantindo independência de servidores externos e facilitando a validação local do protótipo.

O processo de implementação seguiu os princípios do design centrado no usuário \citeonline{norman2013}, priorizando simplicidade, clareza e redução de elementos visuais que possam gerar sobrecarga cognitiva. Tecnologias como Tailwind CSS, Shadcn/UI e Framer Motion foram empregadas para garantir consistência visual, responsividade e transições suaves. A categorização por cores, inspirada em recomendações de design inclusivo, foi utilizada para auxiliar o reconhecimento imediato das informações \citeonline{almeida2024}.

As Figuras~\ref{fig:tela-inicial}, \ref{fig:nova-tarefa} e \ref{fig:detalhes-tarefa} apresentam as três principais telas do aplicativo: Tela Inicial, Nova Tarefa e Detalhes da Tarefa, ilustrando a estrutura visual e os elementos fundamentais da interface.

\begin{figure}[H]
    \centering
    \begin{subfigure}{0.32\textwidth}
        \centering
        \includegraphics[width=0.95\linewidth]{figuras/Tela inicial.png}  
        \caption{Tela inicial}
        \label{fig:tela-inicial}
    \end{subfigure}
    \hfill
    \begin{subfigure}{0.32\textwidth}
        \centering
        \includegraphics[width=0.95\linewidth]{figuras/Nova Tarefa.png}  
        \caption{Nova tarefa}
        \label{fig:nova-tarefa}
    \end{subfigure}
    \hfill
    \begin{subfigure}{0.32\textwidth}
        \centering
        \includegraphics[width=0.95\linewidth]{figuras/Detalhes da Tarefa.png}  
        \caption{Detalhes da tarefa}
        \label{fig:detalhes-tarefa}
    \end{subfigure}
    \caption{Prints do aplicativo TimeX}
    \label{fig:prints}
\end{figure}

Além da navegação simples, o aplicativo conta com notificações via \textit{Web Notifications API} e alertas sonoros pela \textit{Web Audio API},recursos que podem auxiliar usuários que enfrentam dificuldades de foco e organização a manter atenção e lembrar compromissos.

Por fim, todo o código-fonte do \textit{TimeX} foi disponibilizado em repositórios públicos no GitHub: o código da aplicação em \url{https://github.com/MechlerLucas/TimeX-App} e a documentação em \url{https://github.com/MechlerLucas/TimeX-Docs}.

\subsection{Avaliação Heurística de Usabilidade}

Antes da análise heurística, apresenta-se, na Tabela~\ref{tab:lista-heuristicas}, a lista das dez heurísticas propostas por \citeonline{nielsen1994}, que serviram como referência para a avaliação das telas do aplicativo.

\begin{table}[H]
\centering
\caption{Lista das 10 Heurísticas de Nielsen (1994)}
\footnotesize
\label{tab:lista-heuristicas}
\begin{tabular}{|p{1.2cm}|p{13cm}|}
\hline
\textbf{H1} & \textbf{Visibilidade do status do sistema:} Manter o usuário sempre informado sobre o que está acontecendo por meio de feedback apropriado. \\ \hline
\textbf{H2} & \textbf{Correspondência entre o sistema e o mundo real:} Utilizar linguagem, metáforas e elementos familiares ao usuário. \\ \hline
\textbf{H3} & \textbf{Controle e liberdade do usuário:} Permitir que o usuário desfaça ações e retorne sem prejuízo. \\ \hline
\textbf{H4} & \textbf{Consistência e padrões:} Manter uniformidade visual, textual e funcional em toda a interface. \\ \hline
\textbf{H5} & \textbf{Prevenção de erros:} Reduzir a probabilidade de erros e, quando possível, solicitar confirmações antes de ações críticas. \\ \hline
\textbf{H6} & \textbf{Reconhecimento em vez de memorização:} Tornar opções visíveis, minimizando a carga de memória do usuário. \\ \hline
\textbf{H7} & \textbf{Flexibilidade e eficiência de uso:} Atender tanto usuários iniciantes quanto experientes, oferecendo atalhos e uso ágil. \\ \hline
\textbf{H8} & \textbf{Design estético e minimalista:} Evitar excesso de informações, priorizando clareza e foco. \\ \hline
\textbf{H9} & \textbf{Ajuda no diagnóstico e correção de erros:} Mensagens de erro devem ser claras, indicando causas e soluções. \\ \hline
\textbf{H10} & \textbf{Ajuda e documentação:} Disponibilizar documentação clara, breve e acessível, quando necessária. \\ \hline
\end{tabular}
\end{table}

O detalhamento abaixo sintetiza os resultados da análise heurística realizada a partir das diretrizes de \citeonline{nielsen1994}.

\subsection*{Detalhamento das heurísticas por tela}
\label{tab:avaliacao-heuristica}
\subsubsection*{Tela Inicial}

\begin{itemize}
    \item \textbf{H1 – Visibilidade do status:} Presente na exibição da data atual, filtro selecionado e aba ativa na barra inferior.
    \item \textbf{H2 – Correspondência com o mundo real:} Ícones universais como casa, calendário e “+”.
    \item \textbf{H6 – Reconhecimento em vez de memorização:} Tarefas identificadas por cores e filtros visuais no topo.
    \item \textbf{H7 – Flexibilidade e eficiência:} Filtros por cor permitem navegação rápida.
    \item \textbf{H8 – Estética e design minimalista:} Layout limpo, com ênfase nos elementos essenciais.
    \item \textbf{H10 – Ajuda e documentação mínima:} O ícone “+” comunica claramente sua função.
\end{itemize}

\subsubsection*{Tela Nova Tarefa}

\begin{itemize}
    \item \textbf{H1 – Visibilidade do status:} Título “Nova Tarefa” e campos claramente identificados.
    \item \textbf{H3 – Controle e liberdade:} Botão “Cancelar” permite interromper a ação.
    \item \textbf{H4 – Consistência e padrões:} Elementos visuais seguem o padrão da tela inicial.
    \item \textbf{H5 – Prevenção de erros:} Impede salvar sem título e destaca campos obrigatórios.
    \item \textbf{H6 – Reconhecimento:} Seleção de categorias por cores dispensa memorização.
    \item \textbf{H10 – Ajuda mínima:} Ícones e labels indicam claramente a função de cada campo.
\end{itemize}

\subsubsection*{Tela Detalhes da Tarefa}

\begin{itemize}
    \item \textbf{H2 – Correspondência com o mundo real:} Cores e informações que representam a categoria da tarefa.
    \item \textbf{H3 – Controle e liberdade:} Ações de editar, excluir e retornar.
    \item \textbf{H4 – Consistência:} Estrutura visual compatível com as demais telas.
    \item \textbf{H6 – Reconhecimento:} Cor e ícone da categoria facilitam identificação.
    \item \textbf{H8 – Estética minimalista:} Exibe apenas informações essenciais ao usuário.
\end{itemize}


A análise confirmou que o \textit{TimeX} apresenta alto nível de aderência às heurísticas fundamentais: visibilidade do status (H1), metáforas compreensíveis (H2), consistência visual (H4), reconhecimento em vez de memorização (H6) e design minimalista (H8). Esses aspectos são essenciais para reduzir a sobrecarga cognitiva, conforme discutido na literatura sobre TDAH.

\subsubsection{Impressões iniciais de um usuário com TDAH}

Durante o desenvolvimento do protótipo, foram coletadas impressões iniciais de um usuário que realizou um teste exploratório do TimeX. Esse usuário, diagnosticado com TDAH, ofereceu contribuições particularmente relevantes, pois suas percepções dialogam diretamente com o perfil descrito na literatura. Entre os pontos positivos, destacou o layout simples e a divisão de categorias, que facilitaram a visualização das tarefas. Por outro lado, sugeriu que o alarme fosse mais intrusivo, argumentando que lembretes pouco chamativos tendem a passar despercebidos. Outro ponto observado foi a marcação automática de tarefas atrasadas como concluídas, o que dificultou a distinção entre atividades realmente realizadas e atividades pendentes. O usuário sugeriu a existência de um estado “não feita” ou “atrasada”, bem como a possibilidade de reagendar automaticamente tarefas não concluídas para o dia seguinte, recurso presente em outros aplicativos. Contudo, destacou que essa funcionalidade deve considerar tarefas com periodicidade fixa, para evitar inconsistências. Esse feedback reforça aspectos discutidos nas heurísticas e evidencia oportunidades práticas de melhoria no protótipo.


\subsection{Discussão dos Resultados}

Os resultados sugerem que o \textit{TimeX} apresenta potencial para oferecer uma solução simples e direta, alinhada às necessidades descritas na literatura sobre usuários com dificuldades de foco e organização. A combinação entre categorização por cores, organização visual limpa e fluxos reduzidos facilita o foco, diminui distrações e torna o gerenciamento de tarefas mais previsível.

As funcionalidades implementadas dialogam com limitações frequentemente relatadas por pessoas com TDAH, como dificuldade de organização, esquecimento recorrente e necessidade de reforços visuais. A aderência ampla às heurísticas de \citeonline{nielsen1994} reforça a qualidade da interface e evidencia que aplicações minimalistas podem reduzir a carga cognitiva de forma mais eficiente do que ferramentas complexas de produtividade.

Assim, os resultados mostram que é possível conciliar simplicidade técnica, usabilidade e clareza visual em uma solução prática e aplicável. Como trabalhos futuros, recomenda-se a realização de testes com usuários reais para validação empírica e a expansão do sistema com funções opcionais de personalização.



%%%%%%%%%%%%%%%%%%%%%%%%%%%%%%%%%%%%%%%%%%%%%%%%%%%%%%%%%%%%%%%%%%%%%%%%%%%%%%%%%%%%%%%%%%%%%%%%%%%%%%%%%%%%%%%%%%%%%%%%%%%%%%%%

\section{Conclusão}

Este trabalho teve como objetivo desenvolver o aplicativo \textit{TimeX}, um protótipo simples de organização pessoal voltado ao apoio de usuários que apresentam dificuldades de foco e planejamento descritas na literatura sobre TDAH. A proposta buscou demonstrar que aplicativos minimalistas podem ser potencialmente mais adequados para esse perfil de usuário quando comparados a ferramentas tradicionais de produtividade que apresentam excesso de funcionalidades.

O \textit{TimeX} foi implementado com foco na simplicidade visual, na clareza de uso e na redução da carga cognitiva durante as interações. As funcionalidades desenvolvidas, como alarmes configuráveis, categorização por cores e lembretes automáticos, foram projetadas com base em recomendações presentes na literatura, buscando reduzir distrações e tornar o uso mais direto. A interface adota uma abordagem minimalista e intuitiva. A avaliação heurística de usabilidade confirmou que o sistema apresenta boa conformidade com os princípios de design centrado no usuário, indicando que o protótipo é funcional e coerente com os objetivos definidos para esta etapa. Além disso, considerou-se o feedback exploratório de um usuário diagnosticado com TDAH, cujas impressões destacaram tanto aspectos positivos da experiência quanto sugestões de melhoria, reforçando pontos relevantes observados durante a análise.

Os resultados obtidos demonstraram compatibilidade com as expectativas iniciais, indicando que é possível unir simplicidade técnica e clareza visual em uma solução digital que apresenta potencial para apoiar usuários com desafios de organização. Apesar da ausência de testes empíricos com um grupo amplo de usuários, a análise heurística e o feedback exploratório permitiram validar a pertinência do protótipo e identificar oportunidades de aprimoramento, como a inclusão de confirmações de ação e mensagens de feedback mais detalhadas. Tais ajustes poderão tornar a experiência mais completa e robusta em versões futuras.

A principal contribuição desta pesquisa foi o desenvolvimento de uma ferramenta digital que aplica princípios de usabilidade e design minimalista inspirados na literatura sobre TDAH. O \textit{TimeX} representa um avanço inicial ao demonstrar como soluções simples podem ser estruturadas para apoiar usuários com dificuldades de foco e planejamento, contribuindo para a discussão sobre abordagens alternativas aos aplicativos de produtividade tradicionais. Além disso, o projeto evidencia que a aplicação de princípios de design centrado no usuário e heurísticas de usabilidade pode resultar em interfaces mais claras, consistentes e alinhadas às necessidades observadas na literatura.

Como trabalhos futuros, recomenda-se a realização de testes de usabilidade com usuários diagnosticados com TDAH, a fim de validar empiricamente as percepções levantadas na literatura e no feedback exploratório obtido. Sugere-se também o aprimoramento da aplicação com recursos adicionais, como sincronização em nuvem, notificações personalizadas e integração com assistentes virtuais. A continuidade deste projeto pode contribuir significativamente para o desenvolvimento de ferramentas digitais que apoiem a organização pessoal e promovam maior autonomia em contextos reais de uso.



%%%%%%%%%%%%%%%%%%%%%%%%%%%%%%%%%%%
%% FIM DO TEXTO
%%%%%%%%%%%%%%%%%%%%%%%%%%%%%%%%%%%

% \selectlanguage{brazil}
%%%%%%%%%%%%%%%%%%%%%%%%%%%%%%%%%%%
%% Inicio bibliografia
%%%%%%%%%%%%%%%%%%%%%%%%%%%%%%%%%%%

 \newpage
\singlespace{
\renewcommand\refname{REFERÊNCIAS}
\bibliography{bibliografia}
}

%Inclusão do arquivo abntex2-alf.bst como solução para adequação à ABNT NBR 10520:2023 quanto às citações, que não são mais em caixa-alta
\bibliographystyle{abntex2-alf.bst}

\end{document}